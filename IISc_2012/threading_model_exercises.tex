\documentclass[a4paper]{article}

\usepackage{graphics,amsmath,fullpage}
\usepackage[utf8]{inputenc}

\begin{document}

\section{A data-flow runtime}

\begin{enumerate}
\item Download the archive \texttt{dfruntime.tar.gz} from the course's
  web site.  It supports the TStar data-driven execution model
  described in the course.

\item Compile the library and run the test example on a GNU/Linux
  distribution with gcc 4.4 or later; there are a few warnings related
  with GOMP, gcc's OpenMP implementation.

\item Understand as much of the code as possible.
  
  The part of the code with the \texttt{GOMP}-prefixed function calls
  is a rather convoluted and non-portable way to wrap all the
  \texttt{main()} function of a C program into an OpenMP
  \texttt{parallel} region.

  The pragma syntax \texttt{\#pragma omp task} marks the next
  statement/block for concurrent execution as an OpenMP task. Using
  this syntax, the implementation of \texttt{tdecrease()} immediately
  spawns the dependent task when its synchronization counter reaches
  0.  The underlying OpenMP runtime deals with the scheduling of
  independent tasks, whose semantics is very similar to Cilk threads.

\item Starting from the following sequential implementation of the
  merge-sort algorithm, take inspiration from the test example of the
  data-flow runtime to write a parallel data-flow version of the
  merge-sort algorithm.

\begin{verbatim}
extern int garray[200000];

extern int merge(int, int, int, int, int);

int m_sort(int left, int right)
{
  int mid;
  int a, b, d, f;

  if (right <= left)
      return 0;

  mid = (right + left) / 2;
  a = m_sort(left, mid);
  b = m_sort(mid+1, right);

  f = merge(a, b, left, mid+1, right);

  return f;
}

int main ()
{
  int i = 1;
  int size = sizeof (garray)/sizeof(garray[0]);

  i = m_sort (i-1, size-1);
}
\end{verbatim}

\end{enumerate}

\section{Scheduling}

\begin{enumerate}
\item Download the manual and source code for Cilk 5.4.6 from
  \texttt{http://supertech.csail.mit.edu/cilk}

\item Write a parallel Cilk implementation of the matrix product
  realizing $\mathcal{O}(n^3)$ operations but with a critical path of
  $\mathcal{O}(\log n)$.

\item Look at more examples from the \texttt{examples/} directory of
  Cilk 5.4.6.

\item Download the paper on work-stealing by Blumofe and Leiserson
  from the course's web site.

\item Read Section 3 and understand the proofs.

\item Sketch an example of a Cilk program whose sequential execution
  uses a bounded amount of memory, but where an ad-hoc scheduling
  strategy uses an amount of memory proportional to the number of
  instances of a given task.

\item Read the example, discussion and different design choices about
  cactus stacks in Cilk and Intel's TBB:

  \texttt{http://software.intel.com/en-us/articles/the-thorny-problem-of-the-cactus-stack}

\item Vector Fabrics' vfTasks is a simple task library supporting
  inter-task dependences, with a control flow semantics (as opposed to
  the data-driven semantics of the data-flow runtime above):

  \texttt{http://sourceforge.net/projects/vftasks}
\end{enumerate}

\end{document}
